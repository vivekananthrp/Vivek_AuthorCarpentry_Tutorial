\documentclass[]{article}
\usepackage{lmodern}
\usepackage{amssymb,amsmath}
\usepackage{ifxetex,ifluatex}
\usepackage{fixltx2e} % provides \textsubscript
\ifnum 0\ifxetex 1\fi\ifluatex 1\fi=0 % if pdftex
  \usepackage[T1]{fontenc}
  \usepackage[utf8]{inputenc}
\else % if luatex or xelatex
  \ifxetex
    \usepackage{mathspec}
  \else
    \usepackage{fontspec}
  \fi
  \defaultfontfeatures{Ligatures=TeX,Scale=MatchLowercase}
\fi
% use upquote if available, for straight quotes in verbatim environments
\IfFileExists{upquote.sty}{\usepackage{upquote}}{}
% use microtype if available
\IfFileExists{microtype.sty}{%
\usepackage{microtype}
\UseMicrotypeSet[protrusion]{basicmath} % disable protrusion for tt fonts
}{}
\usepackage[margin=1in]{geometry}
\usepackage{hyperref}
\hypersetup{unicode=true,
            pdftitle={Data Management Plan for the Study Do Reputable Open Access Journals Require Open Data Sharing?},
            pdfauthor={R.P. Vivek-Ananth, IMSc, Chennai, India},
            pdfborder={0 0 0},
            breaklinks=true}
\urlstyle{same}  % don't use monospace font for urls
\usepackage{color}
\usepackage{fancyvrb}
\newcommand{\VerbBar}{|}
\newcommand{\VERB}{\Verb[commandchars=\\\{\}]}
\DefineVerbatimEnvironment{Highlighting}{Verbatim}{commandchars=\\\{\}}
% Add ',fontsize=\small' for more characters per line
\usepackage{framed}
\definecolor{shadecolor}{RGB}{248,248,248}
\newenvironment{Shaded}{\begin{snugshade}}{\end{snugshade}}
\newcommand{\KeywordTok}[1]{\textcolor[rgb]{0.13,0.29,0.53}{\textbf{#1}}}
\newcommand{\DataTypeTok}[1]{\textcolor[rgb]{0.13,0.29,0.53}{#1}}
\newcommand{\DecValTok}[1]{\textcolor[rgb]{0.00,0.00,0.81}{#1}}
\newcommand{\BaseNTok}[1]{\textcolor[rgb]{0.00,0.00,0.81}{#1}}
\newcommand{\FloatTok}[1]{\textcolor[rgb]{0.00,0.00,0.81}{#1}}
\newcommand{\ConstantTok}[1]{\textcolor[rgb]{0.00,0.00,0.00}{#1}}
\newcommand{\CharTok}[1]{\textcolor[rgb]{0.31,0.60,0.02}{#1}}
\newcommand{\SpecialCharTok}[1]{\textcolor[rgb]{0.00,0.00,0.00}{#1}}
\newcommand{\StringTok}[1]{\textcolor[rgb]{0.31,0.60,0.02}{#1}}
\newcommand{\VerbatimStringTok}[1]{\textcolor[rgb]{0.31,0.60,0.02}{#1}}
\newcommand{\SpecialStringTok}[1]{\textcolor[rgb]{0.31,0.60,0.02}{#1}}
\newcommand{\ImportTok}[1]{#1}
\newcommand{\CommentTok}[1]{\textcolor[rgb]{0.56,0.35,0.01}{\textit{#1}}}
\newcommand{\DocumentationTok}[1]{\textcolor[rgb]{0.56,0.35,0.01}{\textbf{\textit{#1}}}}
\newcommand{\AnnotationTok}[1]{\textcolor[rgb]{0.56,0.35,0.01}{\textbf{\textit{#1}}}}
\newcommand{\CommentVarTok}[1]{\textcolor[rgb]{0.56,0.35,0.01}{\textbf{\textit{#1}}}}
\newcommand{\OtherTok}[1]{\textcolor[rgb]{0.56,0.35,0.01}{#1}}
\newcommand{\FunctionTok}[1]{\textcolor[rgb]{0.00,0.00,0.00}{#1}}
\newcommand{\VariableTok}[1]{\textcolor[rgb]{0.00,0.00,0.00}{#1}}
\newcommand{\ControlFlowTok}[1]{\textcolor[rgb]{0.13,0.29,0.53}{\textbf{#1}}}
\newcommand{\OperatorTok}[1]{\textcolor[rgb]{0.81,0.36,0.00}{\textbf{#1}}}
\newcommand{\BuiltInTok}[1]{#1}
\newcommand{\ExtensionTok}[1]{#1}
\newcommand{\PreprocessorTok}[1]{\textcolor[rgb]{0.56,0.35,0.01}{\textit{#1}}}
\newcommand{\AttributeTok}[1]{\textcolor[rgb]{0.77,0.63,0.00}{#1}}
\newcommand{\RegionMarkerTok}[1]{#1}
\newcommand{\InformationTok}[1]{\textcolor[rgb]{0.56,0.35,0.01}{\textbf{\textit{#1}}}}
\newcommand{\WarningTok}[1]{\textcolor[rgb]{0.56,0.35,0.01}{\textbf{\textit{#1}}}}
\newcommand{\AlertTok}[1]{\textcolor[rgb]{0.94,0.16,0.16}{#1}}
\newcommand{\ErrorTok}[1]{\textcolor[rgb]{0.64,0.00,0.00}{\textbf{#1}}}
\newcommand{\NormalTok}[1]{#1}
\usepackage{longtable,booktabs}
\usepackage{graphicx,grffile}
\makeatletter
\def\maxwidth{\ifdim\Gin@nat@width>\linewidth\linewidth\else\Gin@nat@width\fi}
\def\maxheight{\ifdim\Gin@nat@height>\textheight\textheight\else\Gin@nat@height\fi}
\makeatother
% Scale images if necessary, so that they will not overflow the page
% margins by default, and it is still possible to overwrite the defaults
% using explicit options in \includegraphics[width, height, ...]{}
\setkeys{Gin}{width=\maxwidth,height=\maxheight,keepaspectratio}
\IfFileExists{parskip.sty}{%
\usepackage{parskip}
}{% else
\setlength{\parindent}{0pt}
\setlength{\parskip}{6pt plus 2pt minus 1pt}
}
\setlength{\emergencystretch}{3em}  % prevent overfull lines
\providecommand{\tightlist}{%
  \setlength{\itemsep}{0pt}\setlength{\parskip}{0pt}}
\setcounter{secnumdepth}{0}
% Redefines (sub)paragraphs to behave more like sections
\ifx\paragraph\undefined\else
\let\oldparagraph\paragraph
\renewcommand{\paragraph}[1]{\oldparagraph{#1}\mbox{}}
\fi
\ifx\subparagraph\undefined\else
\let\oldsubparagraph\subparagraph
\renewcommand{\subparagraph}[1]{\oldsubparagraph{#1}\mbox{}}
\fi

%%% Use protect on footnotes to avoid problems with footnotes in titles
\let\rmarkdownfootnote\footnote%
\def\footnote{\protect\rmarkdownfootnote}

%%% Change title format to be more compact
\usepackage{titling}

% Create subtitle command for use in maketitle
\newcommand{\subtitle}[1]{
  \posttitle{
    \begin{center}\large#1\end{center}
    }
}

\setlength{\droptitle}{-2em}

  \title{Data Management Plan for the Study `Do Reputable Open Access Journals
Require Open Data Sharing'?}
    \pretitle{\vspace{\droptitle}\centering\huge}
  \posttitle{\par}
    \author{R.P. Vivek-Ananth, IMSc, Chennai, India}
    \preauthor{\centering\large\emph}
  \postauthor{\par}
      \predate{\centering\large\emph}
  \postdate{\par}
    \date{Thursday, August 09, 2018}


\begin{document}
\maketitle

\begin{center}\rule{0.5\linewidth}{\linethickness}\end{center}

\begin{Shaded}
\begin{Highlighting}[]
\CommentTok{# Add the dataset of DOAJ Seal journals}

\NormalTok{doaj_seal <-}\StringTok{ }\KeywordTok{read_csv}\NormalTok{(}\StringTok{"data/DOAJ_Seal.csv"}\NormalTok{)}
\end{Highlighting}
\end{Shaded}

\section{Administrative Data}\label{administrative-data}

\subsection{ID}\label{id}

Not Applicable

\subsection{Funder}\label{funder}

This research is being submitted for funding to the J. Bohannon
Foundation \url{http://www.johnbohannon.org/}.

\subsection{Grant Reference Number}\label{grant-reference-number}

Not Applicable (Proposal in preparation)

\subsection{Project Name}\label{project-name}

Do Reputable Open Access Journals Require Open Data Sharing?

\subsubsection{Project Description}\label{project-description}

This study analyzes the submission requirements of the most reputable
open access journals to determine the prevalence and characteristics of
data sharing policies. This question is an important one for 21st
century authors and readers because open data sharing is seen as a key
component of open and more trusted scientific record.

According to the Research Data Alliance (RDA) Data Policy
Standardisation and Implementation group
\url{https://www.rd-alliance.org/groups/data-policy-standardisation-and-implementation}:

``the prevalence of research data policies from institutions and
research funders is increasing, so publishers and editors are paying
more attention to standardisation and the wider adoption of data sharing
policies.''

This study investigates whether the most reputable Open Access journals
have data sharing polices and the characteristics of those policies.
These policies require authors, in some fashion, to openly disseminate
the data and software underlying their published articles.

Our investigation builds on the recent work of Castro et al (2017) who
assessed the prevalence and characteristics of data sharing policies
from randomly-selected, English-language, open access journals. Their
findings reveal that only a small minority of these journals have data
sharing policies. These findings -- which are consistent with those of
other studies {[}see for example, Vasilevsky et al, 2017 -- may be
skewed because of the authors' rules of inclusion and exclusion (FN: in
particular, the choice to include open access journals merely because of
their use of the Open Journal Systems (OJS) hosting platform; the choice
to exclude non\_English language journals).

In this study, we will include only the most reputable open access
journals in our assessment of journal sharing policies, regardless of
language. We will analyze all journals that have attained the Seal of
Approval from the Directory of Open Access Journals, DOAJ
\url{http://doaj.org} (shown below). We will apply the same coding
framework devised by Castro et al (2017) to the DOAJ Seal journals. We
contend that a more rigorously screened population of open access
journals, regardless of language, will yield a more accurate and
reproducible set of findings than those published from Castro et al
(2017)

\textbar{}\textbar{}\textgreater{} Insert DOAJ Seal of Approval
images/doaj\_seal\_logo.png \textless{}\textbar{}\textbar{}

DOAJ Seal journals are considered the most reputable because they:

``achieve a high level of openness, adhere to Best Practice and high
publishing standards.The Seal is awarded to a journal that fulfills a
set of criteria related to accessibility, openness, discoverability,
reuse and author rights. It acts as a signal to readers and authors that
the journal has generous use and reuse terms, author rights and adheres
to the highest level of `openness'.'' (FN: DOAJ selection for Seal
Approval is explained in the FAQ at \url{https://doaj.org/faq\#seal})

Moreover, the DOAJ Seal journals do include over 200 non-English
language journals that merit analysis in this study. Excluding these
from the analysis represents cultural bias that undermines reliable
research. The following plot of DOAJ Seal Journals by Country indicates
the problem.

\includegraphics{Base_2013_day2_in_files/figure-latex/plot_country-1.pdf}

Finally, our research group has determined that following is true when
it comes to reputable open access journals.

\textbar{}\textbar{}\textgreater{} Turn this into a code chunk \{r
equations, child = ``equations-child.Rmd''\}
\textless{}\textbar{}\textbar{}

\subsubsection{Researcher Information}\label{researcher-information}

YourName, Principal Investigator, YourInstitution

Researcher ID

ORCID: \textbar{}\textbar{}\textgreater{} Insert your ORCID number here
\textless{}\textbar{}\textbar{}

Date of First Version

Two months ago

Date of Last Update Today's date

Related Policies

All original data, code, or reports produced as part of this project are
owned by Your Institution per its institutional intellectual property
policy.

The J. Bohannan Foundation adheres to the open access and data sharing
policies of the Gates Foundation
\url{https://www.gatesfoundation.org/How-We-Work/General-Information/Open-Access-Policy}

\begin{center}\rule{0.5\linewidth}{\linethickness}\end{center}

Data Collection

Existing Data Being Reused

This study relies on the DOAJ Journal Metadata available as a csv file
download from the DOAJ website \url{https://doaj.org/faq\#metadata}. The
csv file is updated every 30 minutes.

This data will be read into RStudio, using the tidyverse package (need
to cite Tidyverse somehow) to filter it for those journals awarded the
DOAJ Seal, and to remove unneeded columns containing the web addresses
for journal policies around plagiarism, submission fees, and other urls
not related to this study.

The filtered version of the data set will be exported as a new file
named doaj\_seal.csv for importing into analytical software.

A sample of the doaj\_seal.csv data set is shown below. The complete
data set is available in searchable and broweseable format in the
\protect\hyperlink{annex-table}{Annex} near the end of this document.

\begin{Shaded}
\begin{Highlighting}[]
\NormalTok{knitr}\OperatorTok{::}\KeywordTok{kable}\NormalTok{(}\KeywordTok{head}\NormalTok{(doaj_seal, }\DecValTok{4}\NormalTok{), }\DataTypeTok{caption =} \StringTok{'A Table of the first 4 rows of the DOAJ Seal data.'}\NormalTok{)}
\end{Highlighting}
\end{Shaded}

\begin{longtable}[]{@{}llllllrlllrlll@{}}
\caption{A Table of the first 4 rows of the DOAJ Seal
data.}\tabularnewline
\toprule
JnlTitle & Publisher & PubCountry & Fee & WaiverPolicy & Identifiers &
FirstYear & Language & ReviewProcess & Plagiarism & Sub2Pub & JnlLicense
& AuthorCopyright & DOAJ\_Seal\tabularnewline
\midrule
\endfirsthead
\toprule
JnlTitle & Publisher & PubCountry & Fee & WaiverPolicy & Identifiers &
FirstYear & Language & ReviewProcess & Plagiarism & Sub2Pub & JnlLicense
& AuthorCopyright & DOAJ\_Seal\tabularnewline
\midrule
\endhead
Archives Animal Breeding & Copernicus Publications & Germany & No & Yes
& DOI & 1999 & English & Peer review & Yes & 13 & CC BY & TRUE &
Yes\tabularnewline
Bothalia: African Biodiversity \& Conservation & AOSIS & South Africa &
No & NA & DOI & 2014 & English & Double blind peer review & Yes & 12 &
CC BY & TRUE & Yes\tabularnewline
Geographica Helvetica & Copernicus Publications & Germany & No & Yes &
DOI & 1946 & English, French, German, Italian & Double blind peer review
& Yes & 53 & CC BY & TRUE & Yes\tabularnewline
Hereditas & BioMed Central & United Kingdom & Yes & Yes & DOI & 2005 &
English & Blind peer review & Yes & 6 & CC BY & TRUE &
Yes\tabularnewline
\bottomrule
\end{longtable}

\subsection{Data being collected}\label{data-being-collected}

The doaj\_seal.csv data set currently includes over 1000 reputable open
access journals that we will investigate in this study. The data set
will be copied and enhanced with additional columns, resulting in the
processed data set, doaj\_seal\_enhanced.csv. The following columns will
be added to doaj\_seal.csv, in conformance with the Coding Framework of
Castro et al (2017).

`Data Policy' (Boolean) Yes No

`Data Sharing Policy' (Factor) No mention Implied Mentioned Explicitly
encouraged Required, but not explicitly tied to editorial decisions
Required as a condition of publication

`Data Citation Policy' (Factor) No mention Implied Explicitly encouraged

Investigators will examine the websites of each journal listed in the
doaj\_seal\_enhanced.csv file to determine whether the data sharing
policy is included in the Instructions to Authors. The Coding Framework
published by Castro et al (2017) will be applied.

Data file formats and standards

All data retrieved for the DOAJ Seal of Approval are downloaded and
stored in the open common separated value (csv) file format.

All data policies culled from the web sites of the DOAJ seal journals
will be saved as .txt files. The data generated by applying Castro et
al's (2017) Coding Framework will be stored in csv file format.

Analysis, visualization, and summarization of the study's findings will
be performed in the open source software R and RStudio using the
tidyverse package (cite tidyverse package). Reports produced from the
study will be also be created in RStudio using the open source text
format Rmarkdown (cite rmarkdown package) and output to HTML documents,
slides, and MS Word documents for submission to funders or publishers
(cite Xie et al, 2018, RMarkdown guide from CRC)

All files associated with the project will be maintained under the Git
version control system and made openly available for download from the
Principal Investigator's GitHub repository.

\subsubsection{Expected outputs of the
project}\label{expected-outputs-of-the-project}

\begin{longtable}[]{@{}cllll@{}}
\toprule
\begin{minipage}[b]{0.11\columnwidth}\centering\strut
Output \#\strut
\end{minipage} & \begin{minipage}[b]{0.11\columnwidth}\raggedright\strut
Digital Output\strut
\end{minipage} & \begin{minipage}[b]{0.16\columnwidth}\raggedright\strut
Type\strut
\end{minipage} & \begin{minipage}[b]{0.14\columnwidth}\raggedright\strut
Format,Duration,Size\strut
\end{minipage} & \begin{minipage}[b]{0.18\columnwidth}\raggedright\strut
Planned access\strut
\end{minipage}\tabularnewline
\midrule
\endhead
\begin{minipage}[t]{0.11\columnwidth}\centering\strut
1\strut
\end{minipage} & \begin{minipage}[t]{0.11\columnwidth}\raggedright\strut
doaj\_seal.csv\strut
\end{minipage} & \begin{minipage}[t]{0.16\columnwidth}\raggedright\strut
raw data set downloaded from DOAJ site\strut
\end{minipage} & \begin{minipage}[t]{0.14\columnwidth}\raggedright\strut
CSV file, plain text format, 2.7 MB\strut
\end{minipage} & \begin{minipage}[t]{0.18\columnwidth}\raggedright\strut
\strut
\end{minipage}\tabularnewline
\begin{minipage}[t]{0.11\columnwidth}\centering\strut
2\strut
\end{minipage} & \begin{minipage}[t]{0.11\columnwidth}\raggedright\strut
doaj\_seal\_enhanced.csv\strut
\end{minipage} & \begin{minipage}[t]{0.16\columnwidth}\raggedright\strut
enhanced data set with new data\strut
\end{minipage} & \begin{minipage}[t]{0.14\columnwidth}\raggedright\strut
CSV file, plain text format, 2.7 MB\strut
\end{minipage} & \begin{minipage}[t]{0.18\columnwidth}\raggedright\strut
\strut
\end{minipage}\tabularnewline
\begin{minipage}[t]{0.11\columnwidth}\centering\strut
3\strut
\end{minipage} & \begin{minipage}[t]{0.11\columnwidth}\raggedright\strut
Data set documentation\strut
\end{minipage} & \begin{minipage}[t]{0.16\columnwidth}\raggedright\strut
json metadata file\strut
\end{minipage} & \begin{minipage}[t]{0.14\columnwidth}\raggedright\strut
plain text file, .1 MB\strut
\end{minipage} & \begin{minipage}[t]{0.18\columnwidth}\raggedright\strut
\strut
\end{minipage}\tabularnewline
\begin{minipage}[t]{0.11\columnwidth}\centering\strut
4\strut
\end{minipage} & \begin{minipage}[t]{0.11\columnwidth}\raggedright\strut
Data Processing steps\strut
\end{minipage} & \begin{minipage}[t]{0.16\columnwidth}\raggedright\strut
R scripts and comments\strut
\end{minipage} & \begin{minipage}[t]{0.14\columnwidth}\raggedright\strut
R Notebook file, 1 MB\strut
\end{minipage} & \begin{minipage}[t]{0.18\columnwidth}\raggedright\strut
\strut
\end{minipage}\tabularnewline
\begin{minipage}[t]{0.11\columnwidth}\centering\strut
5\strut
\end{minipage} & \begin{minipage}[t]{0.11\columnwidth}\raggedright\strut
Data Visualizations\strut
\end{minipage} & \begin{minipage}[t]{0.16\columnwidth}\raggedright\strut
R scripts and documentation; Plots\strut
\end{minipage} & \begin{minipage}[t]{0.14\columnwidth}\raggedright\strut
R Notebook file, .png image files 4 MB\strut
\end{minipage} & \begin{minipage}[t]{0.18\columnwidth}\raggedright\strut
\strut
\end{minipage}\tabularnewline
\begin{minipage}[t]{0.11\columnwidth}\centering\strut
6\strut
\end{minipage} & \begin{minipage}[t]{0.11\columnwidth}\raggedright\strut
Journal article\strut
\end{minipage} & \begin{minipage}[t]{0.16\columnwidth}\raggedright\strut
Rendered report\strut
\end{minipage} & \begin{minipage}[t]{0.14\columnwidth}\raggedright\strut
RMarkdown, 9 MB\strut
\end{minipage} & \begin{minipage}[t]{0.18\columnwidth}\raggedright\strut
\strut
\end{minipage}\tabularnewline
\bottomrule
\end{longtable}

\includegraphics{Base_2013_day2_in_files/figure-latex/plot_license-1.pdf}

\section{Documentation and Metadata}\label{documentation-and-metadata}

The journal metadata contained in the project's data sets comes directly
from the Directory of Open Access Journals.

The final outputs from the project will be documented in metadata files
according to the DataCite DOI registration agency -- see the DataCite
Metadata Schema 4.1 \url{https://schema.datacite.org/} for specific
details. By following this standard metadata format, other researchers
(and computers) will be able to find, access, and reuse the outputs from
this project by searching the DataCite metadatabase.

\begin{center}\rule{0.5\linewidth}{\linethickness}\end{center}

Ethics and Legal Compliance

No additional ethical or privacy issues arise in this study because both
the DOAJ data, and the information about data policies for any published
journal, are publicly posted online.

The data provided about journals awarded the Directory of Open Access
Journals Seal of Approval is distributed under a CC BY-SA license. This
license requires that reusers of the data share their derivative data
set under the same license. Therefore, the output of this research will
be disseminated under the CC BY-SA license. This license adheres to the
\emph{Principles and Guidelines of the Research Data Alliance Legal
Interoperability Group}, which recommends the use of Creative Commons
Attribution licenses to allow the broadest sharing of data while
guaranteeing attribution to the data provider. (FN:
\url{https://www.rd-alliance.org/rda-codata-legal-interoperability-research-data-principles-and-implementation-guidelines-now}

\begin{center}\rule{0.5\linewidth}{\linethickness}\end{center}

Storage and Backup

During the active phase of the project data will be stored on and backed
up to the Research Data Storage Facility (RDSF) at My Institution. This
facility represents 2 million pounds of digital resilient storage, with
ongoing capital investment. The RDSF is overseen by a steering group of
senior research and support staff, which includes the PVC Research.
Backup procedures are robust (overnight backup, copies held remotely on
tape) and secured access is in place

\begin{center}\rule{0.5\linewidth}{\linethickness}\end{center}

Selection and long-term preservation

\textbar{}\textbar{}\textgreater{} To be completed by the participants!
\textless{}\textbar{}\textbar{}

\begin{center}\rule{0.5\linewidth}{\linethickness}\end{center}

Data Sharing

The data and metadata will be made openly accessible under a CC-BY SA
license in the CERN-maintained open access repository Zenodo
\url{https://zenodo.org/}, an open dependable home for the long-tail of
science, enabling researchers to share and preserve any research outputs
in any size, any format and from any science. (FN: Zenodo policies are
available online at \url{http://about.zenodo.org/policies/}

\begin{center}\rule{0.5\linewidth}{\linethickness}\end{center}

Responsibilities and Resources

The Principal Investigator is responsible for implementing the Data
Management Plan and ensuring it is reviewed and revised as necessary.
(S)he will be responsible for all data collection and recording; for
data analysis and visualization; and for maintain all files under
version control using git and GitHub.

The Data Management Specialist assigned to the project as an in-kind
contribution from the My Institution Library will be responsible for
creating the DataCite metadata documentation for all outputs and
ensuring timely DOI registration of each final output. (S)he will also
deposit all final outputs to the Zenodo repository and update metadata
associated with the DOI as necessary.

\section*{Annexes}\label{annexes}
\addcontentsline{toc}{section}{Annexes}

\hypertarget{annex-table}{\paragraph{\texorpdfstring{Complete dataset
\texttt{doaj\_seal.csv}}{Complete dataset doaj\_seal.csv}}\label{annex-table}}

\emph{This data table was compiled using the DT package by Xie (2018)}

\textbar{}\textbar{}\textgreater{} Turn this into a code chunk

\texttt{\{r\ data-table\}\ doaj\_seal\ \%\textgreater{}\%\ \ \ datatable(rownames\ =\ FALSE,\ \ \ \ \ \ \ \ \ \ \ \ \ \ colnames\ =\ c("Title",\ "Publisher",\ "Country",\ "Fees",\ "Waivers",\ "Identifiers",\ "Start\ Year",\ "Language(s)",\ "Review\ \ Process",\ "Plagiarism\ check",\ "Time\ to\ Press",\ "License",\ "Author\ owns\ Copyright",\ "DOAJ\ Seal"),\ \ \ \ \ \ \ \ \ \ \ \ \ class\ =\ "cell-border\ stripe",\ \ \ \ \ \ \ \ \ \ \ \ \ \ caption\ =\ "Journals\ with\ DOAJ\ Seal",\ \ \ \ \ \ \ \ \ \ \ \ \ filter\ =\ list(position\ =\ "bottom"),\ \ \ \ \ \ \ \ \ \ \ \ \ extensions\ =\ \textquotesingle{}Buttons\textquotesingle{},\ options\ =\ list(dom\ =\ \textquotesingle{}Bfrtip\textquotesingle{},\ \ \ \ \ \ \ \ \ \ \ \ \ buttons\ =\ c(\textquotesingle{}colvis\textquotesingle{},\ \textquotesingle{}csv\textquotesingle{},\ \textquotesingle{}pdf\textquotesingle{}))\ \ \ )}

\textless{}\textbar{}\textbar{}

\paragraph{Principal Investigator's
BioSketch}\label{principal-investigators-biosketch}

This is auto-populated from your ORCID profile using the rorcid package.
(cite the package).

\textbar{}\textbar{}\textgreater{} insert code from insert\_orcid.R
\textless{}\textbar{}\textbar{}

\section{References}\label{references}

Castro et al, 2017, blahblahblah

Rmarkdown package citation of some kind, blahblahblah

Vasilevsky et al, 2017, blahblahblah

Tidyverse package citation of some kind, blahblahblah

Rorcid package citation of some kind, blahblahblah

Xie (2018), DT Table package, CRAN, blahblahblah

Xie et al (2018) Markdown book published by CRC


\end{document}
